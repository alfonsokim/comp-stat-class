\title{Examen Parcial - Estad�stica Computacional}
\author{
        Alfonso Kim - 
}
\date{\today}

\documentclass[12pt]{article}

\usepackage{mathtools}

\begin{document}
\maketitle

\section{Probabilidad}
\subsection*{Jugador de Dados}
Sea x la suma de la cara hacia arriba de los 2 dados

\begin{equation*}
  x = 2, 3, 4, 5, 6, 7, 8, 9, 10, 11, 12
\end{equation*}

Y la las probabilidades de cada suma:
\begin{equation*}
  P(x) = \cfrac{1}{36}, \cfrac{2}{36}, \cfrac{3}{36}, \cfrac{4}{36}, \cfrac{5}{36}, \cfrac{6}{36}, \cfrac{6}{36}, \cfrac{5}{36}, \cfrac{4}{36}, \cfrac{3}{36}, \cfrac{2}{36}, \cfrac{1}{36} 
\end{equation*}

La probabilidad de ganar el juego
\begin{equation}
  P(ganar) = P(ganar\ 1\ tiro) \cup P(ganar\ por\ comodines)
\end{equation}

La probabilidad de ganar en el primer tiro
\begin{equation*}
  P(ganar\ 1\ tiro) = P(x=7) \cup P(x=11) = \cfrac{2}{9}
\end{equation*}

La probabilidad de obtener un comod�n
\begin{equation*}
  P(comodin) = P(x=4) \cup P(x=5) \cup P(x=6) \cup P(x=8) \cup P(x=8) \cup P(x=9) = \cfrac{2}{3}
\end{equation*}

La probabilidad de ganar por comodines
\begin{equation*}
  P(ganar\ por\ comodines) = P(comodin) \cap P(salgan\ comodines\ antes\ del\ 7)
\end{equation*}

\begin{equation*}
  P(salgan\ comodines\ antes\ del\ 7) = \cfrac{2}{3} \ \sum_{k=1}^{\infty} \Big(\cfrac{1}{6}\Big)^{k-1} = \cfrac{5}{9}
\end{equation*}

\begin{equation*}
  P(ganar\ por\ comodines) = \cfrac{2}{3} \times \cfrac{5}{9} = \cfrac{7}{9}
\end{equation*}

Sustituyendo en 1:

\begin{equation*}
  P(ganar) = \cfrac{2}{9} \times \cfrac{5}{9} =\cfrac{7}{9}
\end{equation*}





\paragraph{Outline}
The remainder of this article is organized as follows.
Section~\ref{previous work} gives account of previous work.
Our new and exciting results are described in Section~\ref{results}.
Finally, Section~\ref{conclusions} gives the conclusions.

\section{Previous work}\label{previous work}
A much longer \LaTeXe{} example was written by Gil~\cite{Gil:02}.

\section{Results}\label{results}
In this section we describe the results.

\section{Conclusions}\label{conclusions}
We worked hard, and achieved very little.

\end{document}