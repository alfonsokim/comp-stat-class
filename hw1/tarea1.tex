\documentclass[letterpaper]{report}

\usepackage[utf8]{inputenc}
\usepackage[spanish]{babel}
\usepackage{listings}
\usepackage{inconsolata}

\usepackage{titlesec}

\renewcommand\thesection{\arabic{section}}
\renewcommand\thesubsection{\thesection.\alph{subsection}}
\renewcommand\thesubsubsection{\thesection.\thesubsection.\alph{subsubsection}}

\makeatletter
\usepackage{graphicx, framed, xcolor}
\usepackage{colortbl}

\usepackage{Sweave}
\begin{document}
\title{Estad\'istica Computacional \\
Tarea 1}
\author{ Itzel M. Fern\'andez de C\'ordova \and Alfonso Kim }
\makeatother

\maketitle

\Sconcordance{concordance:tarea1.tex:tarea1.Rnw:%
1 17 1 1 0 12 1 1 2 1 0 1 1 7 0 1 2 1 1 1 4 3 0 1 1 20 0 1 2 1 1 1 2 1 %
0 3 1 6 0 1 2 1 1 1 2 1 0 2 1 9 0 1 2 1 1 1 2 7 0 1 2 1 1 1 2 7 0 1 2 1 %
1 1 2 1 0 1 1 6 0 1 2 1 1 1 2 7 0 1 2 1 1 1 2 1 0 1 1 7 0 3 1 9 0 2 1 8 %
0 1 2 1 1 1 2 1 0 3 1 7 0 1 1 8 0 1 2 1 1 1 2 1 0 1 1 10 0 4 1 10 0 2 1 %
11 0 1 2 3 1 1 2 1 0 1 5 28 0 1 2 1 1 1 14 13 0 1 1 5 0 1 1 5 0 1 1 7 0 %
1 2 1 1 1 11 10 0 2 1 6 0 1 2 1 1}


\section{Operaciones B\'asicas}

\subsection{}
\begin{Schunk}
\begin{Sinput}
> contarpor5 <- seq(from=5, to=100, by=5)
> contarpor5
\end{Sinput}
\begin{Soutput}
 [1]   5  10  15  20  25  30  35  40  45  50  55  60  65  70  75  80  85  90  95
[20] 100
\end{Soutput}
\end{Schunk}

\subsection{}
\begin{Schunk}
\begin{Sinput}
> Tratamiento <- c(rep("Tratamiento 1", 20), 
+                  rep("Tratamiento 2", 18), 
+                  rep("Tratamiento 3", 22))
> Tratamiento
\end{Sinput}
\begin{Soutput}
 [1] "Tratamiento 1" "Tratamiento 1" "Tratamiento 1" "Tratamiento 1"
 [5] "Tratamiento 1" "Tratamiento 1" "Tratamiento 1" "Tratamiento 1"
 [9] "Tratamiento 1" "Tratamiento 1" "Tratamiento 1" "Tratamiento 1"
[13] "Tratamiento 1" "Tratamiento 1" "Tratamiento 1" "Tratamiento 1"
[17] "Tratamiento 1" "Tratamiento 1" "Tratamiento 1" "Tratamiento 1"
[21] "Tratamiento 2" "Tratamiento 2" "Tratamiento 2" "Tratamiento 2"
[25] "Tratamiento 2" "Tratamiento 2" "Tratamiento 2" "Tratamiento 2"
[29] "Tratamiento 2" "Tratamiento 2" "Tratamiento 2" "Tratamiento 2"
[33] "Tratamiento 2" "Tratamiento 2" "Tratamiento 2" "Tratamiento 2"
[37] "Tratamiento 2" "Tratamiento 2" "Tratamiento 3" "Tratamiento 3"
[41] "Tratamiento 3" "Tratamiento 3" "Tratamiento 3" "Tratamiento 3"
[45] "Tratamiento 3" "Tratamiento 3" "Tratamiento 3" "Tratamiento 3"
[49] "Tratamiento 3" "Tratamiento 3" "Tratamiento 3" "Tratamiento 3"
[53] "Tratamiento 3" "Tratamiento 3" "Tratamiento 3" "Tratamiento 3"
[57] "Tratamiento 3" "Tratamiento 3" "Tratamiento 3" "Tratamiento 3"
\end{Soutput}
\end{Schunk}

\subsection{}
\begin{Schunk}
\begin{Sinput}
> x <- 5
> y <- 7
> z <- x^y
> z
\end{Sinput}
\begin{Soutput}
[1] 78125
\end{Soutput}
\end{Schunk}

\subsection{}
\begin{Schunk}
\begin{Sinput}
> u <- c(1,2,5,4)
> v <- c(2,2,1,1)
> u; v
\end{Sinput}
\begin{Soutput}
[1] 1 2 5 4
\end{Soutput}
\begin{Soutput}
[1] 2 2 1 1
\end{Soutput}
\end{Schunk}

\thesubsubsection{}
\begin{Schunk}
\begin{Sinput}
> which(u == 5)
\end{Sinput}
\begin{Soutput}
[1] 3
\end{Soutput}
\end{Schunk}

\thesubsubsection{}
\begin{Schunk}
\begin{Sinput}
> which(v >= 2)
\end{Sinput}
\begin{Soutput}
[1] 1 2
\end{Soutput}
\end{Schunk}

\thesubsubsection{}
\begin{Schunk}
\begin{Sinput}
> pvec <- c(u[2]*v[3]-u[3]*v[2], u[3]*v[1]-u[1]*v[3], u[1]*v[2]-u[2]*v[1])
> pvec
\end{Sinput}
\begin{Soutput}
[1] -8  9 -2
\end{Soutput}
\end{Schunk}

\thesubsubsection{}
\begin{Schunk}
\begin{Sinput}
> sum(u * v)
\end{Sinput}
\begin{Soutput}
[1] 15
\end{Soutput}
\end{Schunk}

\thesubsubsection{}
\begin{Schunk}
\begin{Sinput}
> X <- matrix(c(u, v), nrow=2, byrow=TRUE) 
> X
\end{Sinput}
\begin{Soutput}
     [,1] [,2] [,3] [,4]
[1,]    1    2    5    4
[2,]    2    2    1    1
\end{Soutput}
\begin{Sinput}
> k <- 2
> Y <- t(X) * k 
> Y
\end{Sinput}
\begin{Soutput}
     [,1] [,2]
[1,]    2    4
[2,]    4    4
[3,]   10    2
[4,]    8    2
\end{Soutput}
\begin{Sinput}
> W <- X %*% Y
> W
\end{Sinput}
\begin{Soutput}
     [,1] [,2]
[1,]   92   30
[2,]   30   20
\end{Soutput}
\end{Schunk}

\thesubsubsection{}
\begin{Schunk}
\begin{Sinput}
> det = (W[1,1] * W[2,2]) - (W[1,2] * W[2,1])
> t <- matrix(c(W[2,2], -W[1,2], -W[2,1], W[1,1]), nrow=2, byrow=TRUE)
> invW <- t * (1 / det)
> invW
\end{Sinput}
\begin{Soutput}
            [,1]        [,2]
[1,]  0.02127660 -0.03191489
[2,] -0.03191489  0.09787234
\end{Soutput}
\begin{Sinput}
> solve(W)
\end{Sinput}
\begin{Soutput}
            [,1]        [,2]
[1,]  0.02127660 -0.03191489
[2,] -0.03191489  0.09787234
\end{Soutput}
\end{Schunk}

\subsection{}
\begin{Schunk}
\begin{Sinput}
> A <- matrix(c(1,2,3,4,5,2,1,2,3,4,3,2,1,2,3,4,3,2,1,2,5,4,3,2,1), nrow=5, ncol=5)
> A
\end{Sinput}
\begin{Soutput}
     [,1] [,2] [,3] [,4] [,5]
[1,]    1    2    3    4    5
[2,]    2    1    2    3    4
[3,]    3    2    1    2    3
[4,]    4    3    2    1    2
[5,]    5    4    3    2    1
\end{Soutput}
\begin{Sinput}
> y <- c(7,-1,-3,5,17)
> invA <- solve(A)
> x <- invA %*% y
> x
\end{Sinput}
\begin{Soutput}
     [,1]
[1,]   -2
[2,]    3
[3,]    5
[4,]    2
[5,]   -4
\end{Soutput}
\begin{Sinput}
> y.t <- A %*% x
> y.t
\end{Sinput}
\begin{Soutput}
     [,1]
[1,]    7
[2,]   -1
[3,]   -3
[4,]    5
[5,]   17
\end{Soutput}
\end{Schunk}

\section{Funciones}

\subsection{}
\begin{Schunk}
\begin{Sinput}
> temps <- seq(from=10, to=100, by=5)
> for (celcius in temps) {
+     fahrenheit <- (9 / 5 * celcius + 32)
+     kelvin <- (celcius - 273)
+     cat(sprintf("%iC = %iF = %iK\n", celcius, fahrenheit, kelvin))
+ }
\end{Sinput}
\begin{Soutput}
10C = 50F = -263K
15C = 59F = -258K
20C = 68F = -253K
25C = 77F = -248K
30C = 86F = -243K
35C = 95F = -238K
40C = 104F = -233K
45C = 113F = -228K
50C = 122F = -223K
55C = 131F = -218K
60C = 140F = -213K
65C = 149F = -208K
70C = 158F = -203K
75C = 167F = -198K
80C = 176F = -193K
85C = 185F = -188K
90C = 194F = -183K
95C = 203F = -178K
100C = 212F = -173K
\end{Soutput}
\end{Schunk}

\subsection{}
\begin{Schunk}
\begin{Sinput}
> magic <- function( a ) {
+     if(a == 1) {
+         s <- c(1)
+     } else if(a == 2) {
+         s <- c(1, 2)
+     } else {
+         s <- c(1, 2)
+         for (i in 3:a) {
+             s <- append(s, s[i-1] + ( 2 / s[i-1] ))
+         }
+     }
+     s
+ }
> magic(1)
\end{Sinput}
\begin{Soutput}
[1] 1
\end{Soutput}
\begin{Sinput}
> magic(2)
\end{Sinput}
\begin{Soutput}
[1] 1 2
\end{Soutput}
\begin{Sinput}
> magic(15)
\end{Sinput}
\begin{Soutput}
 [1] 1.000000 2.000000 3.000000 3.666667 4.212121 4.686941 5.113659 5.504768
 [9] 5.868090 6.208916 6.531033 6.837264 7.129778 7.410292 7.680187
\end{Soutput}
\end{Schunk}

\subsection{}
\begin{Schunk}
\begin{Sinput}
> random.walk <- function(n) {
+     s <- sample(c(-1, 1), n, replace=TRUE)
+     steps <- (0)
+     sum <- 0
+     for (i in 2:length(s)) {
+         sum <- sum + s[i]
+         steps <- append(steps, sum)
+     }
+     steps
+ }
> set.seed(10203040)
> random.walk(10)
\end{Sinput}
\begin{Soutput}
 [1] 0 1 0 1 2 3 2 3 4 3
\end{Soutput}
\end{Schunk}

\end{document}
