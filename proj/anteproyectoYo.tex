\documentclass{article}
\usepackage[spanish]{babel}
\usepackage[utf8]{inputenc}


\author{}
\date{\today}
\title{}

\begin{document}

	\section{Objetivo}
		Mostrar la existencia de una dependencia entre el estado de ánimo de los habitantes de la Ciudad de México y el lugar donde se encuentran dentro de ella a partir de información pública geolocalizada (\emph{Twitter}).

		Nuestro trabajo buscará definir una medida colectiva del humor (positivo, negativo o neutro) para los usuarios de \emph{Twitter} en la Ciudad de México. Planteamos como hipótesis de trabajo que es posible detectar zonas dentro de la ciudad con una medida mayormente positiva (e.g. lugares turísticos, museos, centros comerciales, etc) y lugares donde el sentimiento sea mayormente negativo (e.g. centros de sanciones, delegaciones, entre otros). 


	\section{Contexto Teórico}
		El mundo en el que vivimos actualmente es un mundo globalizado en el cual el uso de las redes sociales crece majestuosamente. Gracias a que hoy en día las noticias se dan a conocer inmediatamente, el público puede responder a éstas al instante mediante estas redes. Las repercusiones de lo que se expresa en ellas y su alcance son impresionantes. Es por esto que se ha vuelto de gran interés el análisis de la información que se puede extraer de las redes sociales.

		\emph{Twitter}, una red social con más de 500 millones de usuarios a nivel mundial, permite la transmisión de mensajes de texto cortos (menos de 140 caracteres), llamados \emph{tweets}, de manera pública o a un grupo selecto de usuarios o seguidores. En esta red es posible proporcionar tu ubicación, y dadas sus características, el contenido de los \emph{tweets} suelen ser opiniones y/o sentimientos que los usuarios piensan y/o sienten en ese preciso momento. Por ello es natural el pensar en una relación entre el estado de ánimo de las personas y su localización dentro de la ciudad. De hecho, muchos estudios se han centrado en analizar el contenido de estos mensajes para inferir los estados de ánimo dentro de una población, obteniendo resultados similares a los de los encuestadores [falta cita].

		El pensar en una relación "humor-espacial" \ es algo natural pues no se experimentan las mismas emociones en un estadio de futból y en un hospital. 

	\section{Datos}
		Usamos el API (Intefaz de Programación de Aplicaciones, por sus siglas en inglés) para extraer los tweets que son generados dentro del Distrito Federal.

		Al día 3 de noviembre del 2013 contamos con más de 5287000 extraidos desde el 18 de junio hasta el 14 de agosto, y desde el 16 de septiembre en adelante, actualmente seguimos extrayendo tweets con un promedio de 49000 tweets diarios. Todos estos contienen la información geográfica de dónde fueron generados en el formato latitud, longitud.

		Es importante notar que, por las condiciones socio-políticas de la Ciudad de México, nuestros datos representan una muestra sesgada de la población, es decir, se captura con mayor probabilidad datos de los estratos de clases sociales alta y media-alta. 

	\section{Método}

\end{document}
